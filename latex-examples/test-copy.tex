\documentclass[a4paper]{article}

\title{Your Paper}
\author{You}

\begin{document}

\section{Introduction}

Your introduction goes here! Some examples of commonly used commands and features are listed below, to help you get started. If you have a question, please use the help menu (``?'') on the top bar to search for help or ask us a question. 

\section{Some examples to get started}

\subsection{How to include Figures}

First you have to upload the image file from your computer using the upload link the project menu. Then use the includegraphics command to include it in your document. Use the figure environment and the caption command to add a number and a caption to your figure. See the code for Figure in this section for an example.

\subsection{How to add Comments}

Comments can be added to your project by clicking on the comment icon in the toolbar above. 
To reply to a comment, simply click the reply button in the lower right corner of the comment, and you can close them when you're done.

Comments can also be added to the margins of the compiled PDF using the todo command, as shown in the example on the right. You can also add inline comments:

\section{Some examples to get started}

\subsection{How to include Figures}

First you have to upload the image file from your computer using the upload link the project menu. Then use the includegraphics command to include it in your document. Use the figure environment and the caption command to add a number and a caption to your figure. See the code for Figure in this section for an example.

\subsection{How to add Comments}

Comments can be added to your project by clicking on the comment icon in the toolbar above. 
To reply to a comment, simply click the reply button in the lower right corner of the comment, and you can close them when you're done.

Comments can also be added to the margins of the compiled PDF using the todo command, as shown in the example on the right. You can also add inline comments:

\subsection{How to add Tables}

Use the table and tabular commands for basic tables --- see Table, for example. 

\subsection{How to write Mathematics}

is great at typesetting mathematics. Let be a sequence of independent and identically distributed random variables with and, and let
denote their mean. Then as approaches infinity, the random variables converge in distribution to a normal.


\subsection{How to create Sections and Subsections}

Use section and subsections to organize your document. Simply use the section and subsection buttons in the toolbar to create them, and we'll handle all the formatting and numbering automatically.

\subsection{How to add Lists}

You can make lists with automatic numbering 

\begin{enumerate}
\item Like this,
\item and like this.
\end{enumerate}
 or bullet points 
\begin{itemize}
\item Like this,
\item and like this.
\end{itemize}

\subsection{How to add Citations and a References List}

You can upload a file containing your BibTeX entries, created with JabRef; or import your, CiteULike or Zotero library as a file. You can then cite entries from it, like this:. Just remember to specify a bibliography style, as well as the filename of the.

You can find a to learn more about BibTeX.

We hope you find Overleaf useful, and please let us know if you have any feedback using the help menu above --- or use the contact form at !

\end{document}